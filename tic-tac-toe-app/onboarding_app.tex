Onboarding – Kundaccept eller avslag med inledande
bedömning
Version 1.0
Redovisningsbyrån AB
25 september 2025

Innehåll
1 Inledning och bakgrund

3

2 Uppdragets syfte och art

3

3 Kundens verksamhetsart
3.1 Ursprung till medel . . . . . . . . . . . . . . . . . . . . . . . . . . . . . . .
3.2 Länder involverade . . . . . . . . . . . . . . . . . . . . . . . . . . . . . . .

4
4
4

4 Kundens identitet
Kontrollöversikt enligt 01FS 2024:20 . . . . . . . . . . . . . . . . . . . . . . . .
4.1 Identifiering och kontroll . . . . . . . . . . . . . . . . . . . . . . . . . . . .
4.2 Kommentar om videomöten och fysiska möten . . . . . . . . . . . . . . . .
4.3 Användning av BankID . . . . . . . . . . . . . . . . . . . . . . . . . . . . .

5
6
8
8
9

5 Verklig huvudman
9
5.1 Aktieägaravtal och kontrollstrukturer . . . . . . . . . . . . . . . . . . . . . 9
5.2 Minimikrav . . . . . . . . . . . . . . . . . . . . . . . . . . . . . . . . . . . 10
5.3 Kopia av aktiebok (obligatoriskt för aktiebolag) . . . . . . . . . . . . . . . 10
6 Sanktioner
11
6.1 Sanktionskontroll . . . . . . . . . . . . . . . . . . . . . . . . . . . . . . . . 11
7 PEP-kontroll (Politically Exposed Person – person i politiskt utsatt
ställning)
12
8 Tidigare byrå

13

9 Bokföringsunderlag och friskrivningsklausul

14

10 Riskbedömning

15

11 Information till kund om penningtvättslagen och förväntade skyldigheter
15
11.1 Avslutande av uppdrag . . . . . . . . . . . . . . . . . . . . . . . . . . . . . 16
1

Onboarding – Kundaccept eller avslag med inledande bedömning,
25 september 2025
12 Avtalsvillkor och godkännande
12.1 Signering . . . . . . . . . . . . . . . . . . . . . . . . . . . . . . . . . . . . .
12.2 Beslut om antagande av kund . . . . . . . . . . . . . . . . . . . . . . . . .
12.3 Motivering beslut . . . . . . . . . . . . . . . . . . . . . . . . . . . . . . . .

17
18
18
19

Onboarding – Kundaccept eller avslag med inledande bedömning,
25 september 2025

1

Inledning och bakgrund

Denna onboarding säkerställer att byrån uppfyller penningtvättslagstiftningens krav vid
antagandet av nya kunder. Processen efterlever tillsynsmyndighetens krav där verksamhetsutövaren (byrån) är skyldig att redovisa hur man säkerställt att byrån inte gör sig
skyldig till penningtvätt.
Länsstyrelserna, som ansvarar för tillsynen, har skärpt kraven på redovisningsbyråer och utfärdar sanktionsavgifter på hundratusentals kronor vid bristande efterlevnad.
Det är dessa myndighetskrav som tvingar oss att ställa specifika frågor och att spara
dokumentationen i minst fem år.
Processen är därmed en följd av myndighetskrav och syftar till att förebygga penningtvätt och ekonomisk brottslighet.

Nästa →

2

Uppdragets syfte och art

Vad exakt är det kunden vill ha hjälp med?
• Typ av uppdrag: Bokföring / Lönehantering / Rådgivning / Deklaration / Annat.
• Frekvens: Löpande / Engångsuppdrag / Periodiskt.
• Eventuella särskilda omständigheter (t.ex. tidsbrist, pågående bokslut).

Nästa →

Onboarding – Kundaccept eller avslag med inledande bedömning,
25 september 2025

3

Kundens verksamhetsart

Penningtvättslagen (PTL) kräver att redovisningsbyråer fastställer varje kunds riskprofil.
Tillsynsmyndigheten Länsstyrelsen Stockholm ställer tydliga krav på att byrån ska kunna
visa att man förstått kundens affärsidé, verksamhetshistorik samt kund- och leverantörsstruktur.
Det är viktigt att kunna bevisa detta vid en eventuell granskning. Om byrån inte
kan styrka att man tydligt övertygat sig om att kundens affärer är rena, kan stora sanktionsavgifter tillfogas – även om kunden själv är helt felfri. Detta kallas "klandervärt
risktagandeöch innebär ansvar för byrån utan att brottspengar behöver vara inblandade.
Se gärna mer information och exempel på faktiska sanktionsbeslut hos Länsstyrelsen
Stockholm.
Frågor som stödjer riskbedömning (flera med lagstöd):
• Vad är företagets huvudsakliga affärsidé?
• Vilka typer av kunder har företaget? Privatpersoner / Företag / Offentlig sektor?
• Har företaget återkommande utländska affärspartners?
• Vilka är de största leverantörerna, och var är de etablerade?
• Har verksamheten ändrats på senare tid?

4

Ekonomisk rådgivning

Denna byrå erbjuder fri ekonomisk rådgivning under onboarding, utan vidare förpliktelser.
För att kunna ge relevanta råd och analysera företagets situation på bästa sätt behöver
vi få tillgång till företagets transaktioner och saldon genom att tilldelas läsrättigheter på
företagets bankkonto.
Detta görs enkelt via våra guider för respektive bank. Lösningen är säker och innebär
att byrån endast har läsbehörighet - vi kan aldrig utföra några transaktioner eller förändra
dina konton. Så snart kopplingen är klar kan vi hämta data, göra en analys och ge konkreta
råd, samt flagga om företaget är berättigat till rabatter eller stöd.
Syftet är att snabbt sätta oss in i företagets ekonomiska förutsättningar och kunna ge
skräddarsydda råd redan från start.
Välj din bank eller myndighet för att komma igång:
• Swedbank (guide)
• Nordea (guide)
• SEB (guide)
• Handelsbanken (guide)
• Danske Bank (guide)
• Skatteverket (deklarationsombud, guide)

Onboarding – Kundaccept eller avslag med inledande bedömning,
25 september 2025
Nästa →

4.1

Ursprung till medel

I enlighet med 3 kap. 3 § penningtvättslagen ska verksamhetsutövaren inhämta information om ursprunget till de medel som används i kundens verksamhet. För etablerade
företag med redovisad verksamhet görs detta genom förståelse av affärsmodellen och finansieringsstrukturen, snarare än genom att spåra enskilda historiska insättningar.
Informationen dokumenteras i fri form i bedömningsunderlaget, exempelvis:
"Verksamheten finansieras uteslutande genom försäljning av tjänster till svenska bolag. Inga externa investerare eller lån förekommer."
Vid komplexa eller högre riskexponerade affärsmodeller bör ytterligare dokumentation
inhämtas, såsom avtal om lån, investeringsavtal eller ägartillskott, särskilt om dessa är av
utländskt ursprung eller saknar tydlig motpart.

4.2

Länder involverade

Förekomst av utländska affärsförbindelser är en viktig faktor vid riskbedömning enligt
penningtvättslagen. Särskild försiktighet ska iakttas om kunden har affärsrelationer med
länder som är listade som högrisk enligt EU:s eller FATF:s aktuella listor, eller med länder
som saknar effektiv tillsynsregim mot penningtvätt och terrorismfinansiering.
Kontrollfrågor
• I vilket land är kunden etablerad?
• Bedrivs verksamhet även i andra länder? Vilka?
• Har kunden affärsförbindelser (kunder/leverantörer/samarbeten) med utlandet? Specificera.
• Förekommer transaktioner till eller från länder utanför EU/EES? Vilka?
• Förekommer något högriskland enligt:
– FATF:s grå och svarta lista?
– EU:s förteckning över högrisktredjeländer?
• Finns exponering mot konfliktområden, sanktionerade jurisdiktioner eller länder med
hög grad av anonymisering (t.ex. offshore-juridik)?

Onboarding – Kundaccept eller avslag med inledande bedömning,
25 september 2025

Rekommenderad åtgärd
Om högriskland identifieras:
• Gör fördjupad riskbedömning.
• Dokumentera varför affärsförbindelsen är motiverad.
• Vid behov: inled inte affärsförhållandet förrän ytterligare information inhämtats.
• Överväg att avvisa uppdraget om förklaring saknas eller om risken är oproportionerlig.

Nästa →

5

Kundens identitet

Fastställandet och kontrollen av kundens identitet är ett krav enligt lagen (2017:630) samt
Länsstyrelsen Stockholms föreskrift 01FS 2024:20, 4 §.
Det är Länsstyrelsen som kräver att vi kan visa bevis på att kontroller har gjorts på
fysiska personer, juridiska företrädare, ombud och verkliga huvudmän. Detta uppfyller vi
genom att bevara kopior av ID-handlingar, bekräftelsebrev eller elektroniska bekräftelser
och tydligt ange när varje kontroll utförts.
Länsstyrelsen kräver att identiteten alltid måste fastställas vid fysisk närvaro eller via
direkt videomöte. Identifiering via telefon eller enbart med BankID utan visuell kontroll
accepterar de inte.

Kontrollöversikt enligt 01FS 2024:20
Nedan hittar du länken till Länsstyrelsens fullständiga krav (skrolla gärna med kunden).
När du har gått igenom dessa går du vidare med de stegvisa checklistorna där vi digitalt
dokumenterar och bifogar handlingar för revision och tillsyn.
Tabell: Identitetskontroll och dokumentation per kundkategori

Onboarding – Kundaccept eller avslag med inledande bedömning,
25 september 2025
Vem kontrolleras
Fysisk person

Vad kontrolleras

Dokumentation av kontrollen

• ID-handling med foto, • Anteckna ID-handlingens numt.ex. pass, körkort eller
mer och giltighetstid eller
nationellt ID
• Bevara en kopia av ID• Alternativt:
tillförlitlig
handlingen eller elektronisk
elektronisk legitimation
bekräftelse
• Vid behov: flera oberoende
källor
Fysisk person på distans

• Elektronisk
eller

legitimation,

• Kopia av elektronisk legitimation eller

• Namn,
personnummer, • Kopia av bekräftelsebrev eller
adress – kontrollerat mot
vidimerad ID-kopia
externa register
• Därefter:
– Bekräftelsebrev till folkbokföringsadress, eller
– Vidimerad
ID-kopia
från tredje part
Företrädare, ombud eller motsva• Se kontroll för fysisk perrande
son eller distans
• Fullmakt, förordnande eller motsvarande

• Kopia av ID-handling
• Kopia av fullmakt/förordnande

Juridisk person
• Registreringsbevis eller re- • Bevara kopia av kontrollerade
gisterutdrag
handlingar
• Alternativt: uppgifter från
andra tillförlitliga källor
Verklig huvudman
• Samma kontroller som för • Se dokumentationskrav för fyfysisk person eller fysisk
sisk person
person på distans

Onboarding – Kundaccept eller avslag med inledande bedömning,
25 september 2025
Kommentar till tabellen
Samtliga kontroller ska dokumenteras vid varje onboardingtillfälle, och av dokumentationen ska det framgå när kontrollen utfördes. Även befintliga kunder kan
behöva genomgå ny kontroll vid förändrade förhållanden.

5.1

Identifiering och kontroll

• Företagsnamn och organisationsnummer: ____________
• Företrädare: Namn, personnummer och befattning.
• Genomförd ID-kontroll: Ja/Nej
• Typ av ID-handling: Pass / Nationellt ID / Körkort / Annat: ___
• Fotodokumentation av ID: Ja / Nej
• Verifikation av fysisk närvaro: Ja / Nej
• Typ av möte: Fysiskt / Videomöte (Zoom, Skype etc.)
• Säkerhetsåtgärder vid distansmöte: Skärmdump / Inspelning / Avvisat p.g.a. osäker identitet
• Kund informerad om lagring i 5 år: Ja / Nej

5.2

Kommentar om videomöten och fysiska möten

Länsstyrelsen ställer tydliga krav på hur identifiering ska göras, och dessa är vi båda
skyldiga att följa. För att undvika sanktioner från Länsstyrelsen krävs att vid videomöten
följande åtgärder genomförs:
• ID-handling ska visas tydligt under samtalet.
• En skärmdump tas där både kundens ansikte och ID-handling syns, som dokumentation.
• Kunden kan bli ombedd att skicka in en separat bild på samma ID-handling.
• Vi informerar om att all dokumentation sparas i fem år enligt gällande lagstiftning.
• Om möjligt dokumenteras tidpunkt och IP-adress för samtalet.
Dessa åtgärder är nödvändiga för att vi ska kunna följa gällande regelverk och skydda
båda parter från eventuella sanktionsavgifter.

Onboarding – Kundaccept eller avslag med inledande bedömning,
25 september 2025

5.3

Användning av BankID

BankID är en godkänd metod för elektronisk identifiering som ofta används vid distansverifiering. Enligt kraven från Länsstyrelsen finns dock vissa risker som måste beaktas:
• Identiteten kan inte fullständigt kopplas till personen utan samtidig visuell kontakt.
• BankID kan användas av någon annan än den verkliga kunden, till exempel om mobilen
lånas ut.
• Därför bör BankID ses som ett komplement och inte som enda metod, särskilt om
verklig huvudman inte bekräftats separat.
Vid s.k. ”hög risk”, det vill säga i de situationer där Länsstyrelsen bedömer att risken
är förhöjd, eller vid nya kundrelationer, kräver myndigheten fysiskt möte. Vid distansverifiering måste BankID därför kombineras med videomöte och skriftlig bekräftelse.

Nästa →

6

Verklig huvudman [Appen hämtar data från Bolagsverket automatiskt]

Du ska utreda om kunden har en eller flera verkliga huvudmän samt genomföra identitetskontroll av dessa. Tillsammans går vi igenom viktiga delar av identifikationen av verkliga
huvudmän. Detta är ett krav enligt lag och viktigt för att säkerställa transparens och
lagenlig efterlevnad.
En verklig huvudman är en fysisk person som – ensam eller tillsammans med andra
– ytterst äger eller kontrollerar kunden. Det första steget är att söka i Bolagsverkets
register över verkliga huvudmän.
Observera att det inte alltid räcker att utgå från vem som äger aktierna. Även personer
som genom avtal, t.ex. aktieägaravtal eller andra arrangemang, utövar faktisk kontroll över
företaget kan vara verkliga huvudmän.

6.1

Aktieägaravtal och kontrollstrukturer

Eftersom aktieägaravtal inte är offentliga, bör du – särskilt vid s.k. "hög risk"– noggrant
undersöka förekomst av sådana avtal. I vissa fall kan kontrollinformationen i stället framgå av bolagsordningen (för aktiebolag) eller av stadgarna (för föreningar), som båda är
offentliga handlingar.

Onboarding – Kundaccept eller avslag med inledande bedömning,
25 september 2025
□ Finns ett aktieägaravtal, dvs. ett papper som reglerar vem som egentligen äger aktierna?
(Kunden svarar på heder och samvete, Ja/Nej)

6.2

Minimikrav

En sökning i Bolagsverkets register över verkliga huvudmän är ett obligatoriskt minimikrav. Denna kontroll ska alltid genomföras, även vid låg risk.
Fördjupad utredning
Om riskprofilen motiverar det, ska ytterligare åtgärder vidtas. Det kan innebära att
uppgifterna i registret måste bekräftas eller kompletteras genom egen utredning. Exempelvis kan du behöva kartlägga ägarstrukturen och begära dokumentation som verifierar
verklig kontroll.
Utebliven registrering av verklig huvudman är i sig inte avgörande men kan i vissa fall
vara en indikation på förhöjd risk. Även frekventa förändringar i registrerade huvudmän
bör föranleda frågor.
• Utredd: Ja / Nej
• Namn på huvudman / huvudmän:
• Kontroll genom: Bolagsverket / Egen utredning (Beskriv hur kontrollen gjorts)

6.3

Kopia av aktiebok (obligatoriskt för aktiebolag)

För aktiebolag ska en uppdaterad kopia av aktieboken alltid begäras in i samband med
kundkännedom. Den används som underlag för att identifiera verklig huvudman samt
upptäcka ägarförändringar.
• Aktiebok mottagen: Ja / Nej
• Senast uppdaterad: __________
• Innehåller fullständig information om:
– Samtliga aktieägare
– Antal aktier per person
– Eventuella överlåtelser och datum
– Namn och personnummer / organisationsnummer
Kommentar:
Om aktieboken saknas, är föråldrad eller ofullständig ska detta dokumenteras och
utgöra en indikator på ökad risk enligt penningtvättslagen. Kontakta bolagets ledning för att begära en uppdaterad version.

Onboarding – Kundaccept eller avslag med inledande bedömning,
25 september 2025
Nästa →

7

Sanktioner

[Denna section visas ej för kunden. Kontrollen sköts i bakgrunden genom förfrågningar
via api. Grönt eller rött ljus ges.]
I samband med kundkännedom ska det alltid kontrolleras om kunden, den verkliga huvudmannen eller företrädare omfattas av ekonomiska eller andra internationella sanktioner.
Sådana sanktioner kan beslutas av:
• Europeiska unionen (EU)
• Förenta Nationerna (FN)
• Utrikesdepartementet i Sverige (UD)
Kontrollen är särskilt viktig vid koppling till högriskländer, politiskt utsatta personer
(PEP) eller verksamhetstyper som enligt den nationella riskbedömningen är förknippade
med hög penningtvättsrisk.

7.1

Sanktionskontroll

[Dokument ifylles automatiskt efter automatisk kontroll]
• Kontroll i sanktionsdatabas genomförd: Ja / Nej
• Resultat:
• Kontrollerade namn:
• Datum för kontroll:
Exempel på sanktionsdatabaser att kontrollera:
• EU:s sanktionslista: https://www.sanctionsmap.eu

• FN:s sanktionslista: https://www.un.org/securitycouncil/sanctions/un-sc-consolidated-lis
• Utrikesdepartementets sanktionsinformation: https://www.regeringen.se/sanktioner
[Visas ej]
Kommentar:
Om kunden eller verklig huvudman förekommer i någon sanktionslista ska detta
omedelbart dokumenteras och utredas. I vissa fall måste uppdraget avbrytas eller
underrättelse skickas till Finanspolisen.

Onboarding – Kundaccept eller avslag med inledande bedömning,
25 september 2025
Nästa →

8

PEP-kontroll (Politically Exposed Person – person i
politiskt utsatt ställning)

Personer i politiskt utsatt ställning (PEP) innehar eller har innehaft offentliga funktioner
med betydande inflytande. Detta medför ökad risk för missbruk, såsom korruption och
penningtvätt.
Enligt 3 kap. 19–20 §§ i Lag (2017:630) om åtgärder mot penningtvätt och finansiering
av terrorism krävs skärpta åtgärder för kunder, verkliga huvudmän eller närstående som
är PEP för att säkerställa affärsrelationens legitimitet.
PEP-status kvarstår minst 18 månader efter att personen lämnat sin offentliga tjänst.

Behandling av personuppgifter
Redovisningsbyrån hanterar personuppgifter i enlighet med gällande dataskyddslagstiftning (GDPR). Kunden informeras om att uppgifter samlas in och behandlas för att fullgöra uppdraget och uppfylla lagstadgade skyldigheter enligt bl.a. penningtvättslagen och
bokföringslagen.
Personuppgifterna sparas endast så länge som krävs för uppdragets genomförande och
lagstadgade arkiveringstider, och skyddas med lämpliga säkerhetsåtgärder. Kunden har
rätt att begära information om vilka uppgifter som lagras och kan kontakta byrån för
frågor kring dataskydd.
Genom att godkänna detta avtal samtycker kunden till denna behandling av personuppgifter.

Kontrollfrågor (dokumenteras)
• Är kunden, företrädaren eller den verkliga huvudmannen PEP? . . . . . . . . . . . . . Ja / Nej
• Om indirekt koppling – vilken? . . . . . . . . . . . . . . . . . Familjemedlem / Känd medarbetare
• Har godkännande inhämtats från behörig beslutsfattare? . . . . . . . . . . . . . . . . . . . . Ja / Nej
• Bedömning av tillgångarnas ursprung: . . . . . . . . . . . . . . . . . . . . . . .
• Skärpt övervakning inledd? . . . . . . . . . . . . . . . . . . . . . . . . . . . . . . . . . . . . . . . . . . . . . . . . . Ja / Nej

Definitioner enligt 1 kap. 9–10 §§
• Politisk utsatthet (PEP): Exempelvis statsråd, parlamentsledamöter, domare, ambassadörer, höga officerare, partiledare och styrelseledamöter i statsägda bolag.

Onboarding – Kundaccept eller avslag med inledande bedömning,
25 september 2025
• Familjemedlemmar: Make/maka, sambo, registrerad partner, barn (och deras makar/partners), samt föräldrar.
• Kända medarbetare: Personer med nära affärsrelationer eller gemensamt ägande med
PEP.

Skärpta åtgärder krävs
Om kunden, verklig huvudman eller närstående är PEP ska alltid följande genomföras:
• Undersökning av tillgångarnas ursprung.
• Fortlöpande uppföljning av affärsrelationen.
• Godkännande från behörig beslutsfattare innan affärsförbindelsen inleds.

Nästa →

9

Tidigare byrå

När en kund tidigare anlitat en annan redovisningsbyrå rekommenderar vi, enligt god sed
(FAR:s vägledning för byråbyte), att den nya byrån visar särskild hänsyn och dokumenterar följande:
• Namn på tidigare byrå eller redovisningskonsult.
• Datum för senaste uppdrag eller samarbete.
• Orsak till byråbyte – dokumentera så exakt som möjligt vad kunden uppger.
• Kontroll av att övergången inte strider mot tidigare uppdragsavtal.
• Om möjligt, inhämta yttrande från tidigare konsult, med kundens medgivande.
Viktigt att notera:
• Kontakta gärna tidigare byrå (med kundens samtycke) för att säkerställa att inga
oklarheter finns.
• Var särskilt uppmärksam om det finns misstankar om bristande dokumentation
eller oredovisade transaktioner.
• Om tidigare konsult inte kan eller vill lämna uppgifter, dokumentera detta noggrant.

Onboarding – Kundaccept eller avslag med inledande bedömning,
25 september 2025
Exempel:
En kund kan uppge att den tidigare konsulten inte levt upp till förväntningarna,
men samtidigt vägra lämna underlag eller möjliggöra kontakt med den tidigare
byrån. Detta kan vara en indikation på problem, särskilt om bokföringen verkar
ofullständig.

Nästa →

10

Bokföringsunderlag och friskrivningsklausul

För att fastställa kundens riskprofil enligt penningtvättslagen behöver vi dokumentera
en god förståelse av företagets affärsidé, verksamhetshistorik samt kund- och leverantörsstruktur.
I detta ingår att vi ber att få ta del av följande underlag för en första granskning:
• Tre senaste årsredovisningarna.
• Tre senaste SIE4-filerna.
Under onboardingprocessen använder vi en friskrivningsklausul som ger oss rätt att
ta del av dessa filer. Det innebär att vi gör denna första bedömning tillsammans med
kunden för att kunna identifiera eventuella risker.
Tillsynsmyndigheten Länsstyrelsen Stockholm ställer tydliga krav på att byrån ska
kunna visa att man förstått kundens affärsidé, verksamhetshistorik samt kund- och leverantörsstruktur.
Det är viktigt att kunna bevisa detta vid eventuell granskning. Om byrån inte kan
styrka att man tydligt övertygat sig om att kundens affärer är rena, kan stora sanktionsavgifter tillfogas – även om kunden själv är helt felfri. Om vi inte förstår bokföringen från
tidigare redovisningsbyrå så måste vi avbryta ett övertagande av uppdraget, så att ingen
part behöver ta på sig onödiga risker.

Nästa →

Onboarding – Kundaccept eller avslag med inledande bedömning,
25 september 2025

11

Riskbedömning

Riskbedömningen är en del av det lagstadgade krav som tillsynsmyndigheten ställer på
verksamhetsutövaren för att bidra till att förebygga penningtvätt och finansiering av terrorism. Syftet är att identifiera och värdera risken att kunden eller kundens verksamhet
kan utnyttjas i olagliga syften.
Denna bedömning baseras på en algoritmisk metodik som följer nationella riktlinjer
och internationella rekommendationer, där faktorer som kundens affärsmodell, geografiska
verksamhetsområden, ägarstruktur, transaktionsmönster och tjänsteutbud vägs in.
Det är verksamhetsutövarens ansvar att genomföra och dokumentera riskbedömningen. Risknivån ska anges som Låg, Normal eller Hög, baserat på en sammanvägd bedömning av samtliga relevanta faktorer.
Byråchefen ska skriva en tydlig och ärlig motivering i fri prosa som förklarar varför
kunden tilldelats den aktuella risknivån och ange särskilda faktorer som påverkat bedömningen. Denna motivering är främst till för att uppfylla tillsynsmyndighetens krav men
används även som underlag i den interna rapporteringen.
Att kunden bedöms ha hög risk betyder inte att kunden är misstänkt eller har begått
brott, utan att det finns omständigheter som kräver extra försiktighet enligt lag.
Riskbedömningen är ett levande dokument som ska ses över och uppdateras vid förändringar i kundens verksamhet eller när ny information framkommer.
(Formulera riskbedömningen i prosa)

Nästa →

12

Information till kund om penningtvättslagen och
förväntade skyldigheter

Kunden har vid inledningen av affärsförbindelsen erhållit muntlig och/eller skriftlig information om vad som förväntas enligt tillämplig lagstiftning, främst Lag (2017:630) om
åtgärder mot penningtvätt och finansiering av terrorism (PTL), samt andra relevanta
regelverk.

Genomgångens innehåll
• Kortfattad introduktion till penningtvättslagen och dess syfte.
• Översikt av redovisningsbyråns skyldigheter enligt PTL samt föreskrifter från Länsstyrelsen och Finansinspektionen.

Onboarding – Kundaccept eller avslag med inledande bedömning,
25 september 2025
• Information om fakturering och fakturors utformning enligt mervärdesskattelagen (2023:200)
17 kap. 15-28 §§.
• Genomgång av skyldigheten att inhämta kundkännedom enligt 3 kap. PTL: ställa frågor om kundens identitet, verksamhet och medlens ursprung, dokumentera kundens
svar och beteende, samt att, vid skälig misstanke, rapportera till Finanspolisen (Finanspolisens underrättelseenhet). Rapporteringen är sekretessbelagd och byrån får inte
informera kunden om att en rapport har gjorts (tipping-off-förbud).
• Förklaring av byråns straffrättsliga och administrativa ansvar vid bristande efterlevnad,
inklusive risk för sanktionsavgifter, tillsynsåtgärder och i vissa fall straffrättsligt ansvar.
Länsstyrelsen har befogenhet att genomföra inspektioner och ingripa vid behov.
• Information om kundens straffrättsliga ansvar: även den som utan uppsåt hanterar
pengar med misstänkt ursprung kan dömas för penningtvätt. Straffansvar kan även uppkomma vid näringspenningtvätt, klandervärt risktagande, likgiltighet eller grov oaktsamhet, samt vid penningtvättsförseelse (ringa brott).
• Betoning av kundens skyldighet att lämna korrekta uppgifter och inkomma med underlag i god tid.
• Information om att varningar kan utfärdas vid upprepade brister eller passivitet.
• Tydlig information om att samarbetet kan avslutas vid avtalsbrott, ovilja att medverka till lagstadgad efterlevnad eller annat väsentligt åsidosättande, se vidare avsnittet
Avslutande av uppdrag. Detta regleras även i uppdragsavtalet och stöds av allmänna
avtalsrättsliga principer enligt Avtalslagen (1915:218).

12.1

Avslutande av uppdrag

Kunden informeras uttryckligen om att samarbetet kan avslutas om denne:
• upprepade gånger inte inkommer med begärda underlag i tid,
• undviker att besvara frågor kopplade till kundkännedom eller verksamhetens art,
• lämnar oriktiga uppgifter eller undanhåller relevant information,
• uppvisar ovilja att förstå eller följa grundläggande redovisningsskyldigheter (t.ex. att
samtliga styrelseledamöter ska underteckna årsredovisningen),
• uppvisar ett beteende som väcker oro för att uppdraget kan främja penningtvätt, eller
• i övrigt bryter mot uppdragsavtalet eller mot lagstadgade skyldigheter.
Beslut om avslutande dokumenteras alltid skriftligt. I tillämpliga fall görs anmälan till
Finanspolisen om det finns skäl att misstänka penningtvätt eller finansiering av terrorism
enligt 4 kap. 3 § PTL. Om lämpligt kan beslutet även kortfattat motiveras skriftligen till
kunden, i syfte att stärka transparens och spårbarhet.

Onboarding – Kundaccept eller avslag med inledande bedömning,
25 september 2025
Notering:
Denna information är även införlivad i uppdragsavtalet och kommuniceras till kunden redan vid onboarding.

Nästa →

13

Avtalsvillkor och godkännande

Genom att underteckna detta dokument godkänner kunden att ingå uppdrag med redovisningsbyrån enligt de villkor som anges nedan. Detta dokument utgör både onboarding
och avtal.

Behandling av personuppgifter
Redovisningsbyrån hanterar personuppgifter i enlighet med gällande dataskyddslagstiftning (GDPR). Kunden informeras om att uppgifter samlas in och behandlas för att fullgöra uppdraget och uppfylla lagstadgade skyldigheter enligt bl.a. penningtvättslagen och
bokföringslagen.
Personuppgifterna sparas endast så länge som krävs för uppdragets genomförande och
lagstadgade arkiveringstider, och skyddas med lämpliga säkerhetsåtgärder. Kunden har
rätt att begära information om vilka uppgifter som lagras och kan kontakta byrån för
frågor kring dataskydd.
Genom att godkänna detta avtal samtycker kunden till denna behandling av personuppgifter.

Uppdragets omfattning
Byrån ska tillhandahålla tjänster enligt vad som framgår av den separat bifogade tjänstebeskrivningen och offert. Kunden förbinder sig att samarbeta fullt ut och tillhandahålla
all nödvändig information i rätt tid.
• Uppdragets tidsperiod: [Ange start- och slutdatum eller att uppdraget gäller tillsvidare]
• Uppdragets omfattning: [Specifik beskrivning av tjänster, t.ex. löpande bokföring,
moms, löner, årsredovisning]
• Pris och betalningsvillkor: [Pris exkl. moms, faktureringsintervall, betalningstid]

Onboarding – Kundaccept eller avslag med inledande bedömning,
25 september 2025

Avtalstid och uppsägning
Avtalet gäller tillsvidare med en minsta bindande period om [exempelvis 12 månader] från
datum för undertecknandet. Uppsägning kan ske med [exempelvis 3 månaders] skriftligt
varsel och träder i kraft först efter bindande periodens utgång.
Byrån förbehåller sig rätten att säga upp avtalet inom en vecka efter att kompletta
uppsättningen av SIE-filer (7 år) erhållits, förutsatt att bokföringen är komplett och
begriplig.

Kundens skyldigheter
Kunden är skyldig att:
• Lämna korrekta, fullständiga och aktuella uppgifter till byrån.
• Omgående rapportera alla förändringar i verksamheten eller annan relevant information
som kan påverka avtalets genomförande eller riskbedömning.
• Försäkra att äldre bokföringsunderlag sparas enligt gällande lagstiftning och finns tillgängliga för byrån eller myndigheter på begäran.
Felaktiga eller vilseledande uppgifter, underlåtenhet att informera om väsentliga förändringar eller bristande samarbete kan leda till uppsägning av avtalet och anmälan till
Finanspolisen i enlighet med gällande regler.
Kunden bekräftar härmed sitt fulla ansvar för ovanstående åtgärder och förstår att
brott mot dessa skyldigheter kan få allvarliga konsekvenser för både kunden själv och
byrån.

13.1

Signering

Genom sin underskrift bekräftar kunden att ha tagit del av och godkänner ovanstående
villkor.
• Kundens namn (text):

Datum:

• Kundens elektroniska underskrift: Verifierad via BankID och säkert dokumenterad
i systemet
• Byråchefens namn (text):

Datum:

• Byråchefens underskrift: Verifierad via BankID och säkert dokumenterad i systemet

13.2

Beslut om antagande av kund

Beslutet dokumenteras genom elektronisk signering med BankID av beslutsfattare och
sparas säkerställt i systemet.
• Beslut fattat av: Om elektronisk signatur via BankID registrerad i systemet Datum:
• Byråchefens namns namn (text):

Datum:

• Beslut om antagande: Kunden antas / antas inte som uppdragsgivare

Onboarding – Kundaccept eller avslag med inledande bedömning,
25 september 2025

13.3

Motivering beslut

[Textfält där byråchef anger motivering för beslutet, sparas som del i dokumentationen]
[Här ska byråchefen skriva en tydlig motivering till beslutet, där skälen för antagande
eller avslag framgår på ett konkret och sakligt sätt.]

